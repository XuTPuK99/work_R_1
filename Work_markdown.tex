% Options for packages loaded elsewhere
\PassOptionsToPackage{unicode}{hyperref}
\PassOptionsToPackage{hyphens}{url}
%
\documentclass[
]{article}
\usepackage{amsmath,amssymb}
\usepackage{lmodern}
\usepackage{iftex}
\ifPDFTeX
  \usepackage[T1]{fontenc}
  \usepackage[utf8]{inputenc}
  \usepackage{textcomp} % provide euro and other symbols
\else % if luatex or xetex
  \usepackage{unicode-math}
  \defaultfontfeatures{Scale=MatchLowercase}
  \defaultfontfeatures[\rmfamily]{Ligatures=TeX,Scale=1}
\fi
% Use upquote if available, for straight quotes in verbatim environments
\IfFileExists{upquote.sty}{\usepackage{upquote}}{}
\IfFileExists{microtype.sty}{% use microtype if available
  \usepackage[]{microtype}
  \UseMicrotypeSet[protrusion]{basicmath} % disable protrusion for tt fonts
}{}
\makeatletter
\@ifundefined{KOMAClassName}{% if non-KOMA class
  \IfFileExists{parskip.sty}{%
    \usepackage{parskip}
  }{% else
    \setlength{\parindent}{0pt}
    \setlength{\parskip}{6pt plus 2pt minus 1pt}}
}{% if KOMA class
  \KOMAoptions{parskip=half}}
\makeatother
\usepackage{xcolor}
\usepackage[margin=1in]{geometry}
\usepackage{graphicx}
\makeatletter
\def\maxwidth{\ifdim\Gin@nat@width>\linewidth\linewidth\else\Gin@nat@width\fi}
\def\maxheight{\ifdim\Gin@nat@height>\textheight\textheight\else\Gin@nat@height\fi}
\makeatother
% Scale images if necessary, so that they will not overflow the page
% margins by default, and it is still possible to overwrite the defaults
% using explicit options in \includegraphics[width, height, ...]{}
\setkeys{Gin}{width=\maxwidth,height=\maxheight,keepaspectratio}
% Set default figure placement to htbp
\makeatletter
\def\fps@figure{htbp}
\makeatother
\setlength{\emergencystretch}{3em} % prevent overfull lines
\providecommand{\tightlist}{%
  \setlength{\itemsep}{0pt}\setlength{\parskip}{0pt}}
\setcounter{secnumdepth}{-\maxdimen} % remove section numbering
\XeTeXdefaultencoding utf8
\usepackage{xltxtra}
\usepackage{graphicx}
\usepackage[multidot]{grffile}
\usepackage{fontspec}
\setmainfont{Times New Roman}
\setsansfont{Arial}
\setmonofont{Courier New}
\newfontfamily{\cyrillicfont}{Times New Roman}
\newfontfamily{\cyrillicfonttt}{Courier New}
\newfontfamily{\cyrillicfontsf}{Arial}
\ifLuaTeX
  \usepackage{selnolig}  % disable illegal ligatures
\fi
\IfFileExists{bookmark.sty}{\usepackage{bookmark}}{\usepackage{hyperref}}
\IfFileExists{xurl.sty}{\usepackage{xurl}}{} % add URL line breaks if available
\urlstyle{same} % disable monospaced font for URLs
\hypersetup{
  pdftitle={Домашнее задание},
  pdfauthor={Тюрнев Иван},
  hidelinks,
  pdfcreator={LaTeX via pandoc}}

\title{Домашнее задание}
\author{Тюрнев Иван}
\date{Декабрь 2022}

\begin{document}
\maketitle

\href{https://github.com/XuTPuK99/work_R_1}{Мой GitHub}

\hypertarget{ux437ux430ux434ux430ux43dux438ux435-ux432ux438ux434ux44b-ux430ux43dux430ux43bux438ux437ux430-ux438-r-markdown.}{%
\section{Задание: Виды анализа и R
Markdown.}\label{ux437ux430ux434ux430ux43dux438ux435-ux432ux438ux434ux44b-ux430ux43dux430ux43bux438ux437ux430-ux438-r-markdown.}}

\hypertarget{ux431ux44bux43b-ux432ux44bux434ux430ux43d-ux432ux430ux440ux438ux430ux43dux442-ux441ux43e-ux441ux43bux435ux434ux443ux44eux449ux438ux43cux438-ux430ux43dux430ux43bux438ux437ux430ux43cux438}{%
\paragraph{Был выдан вариант со следующими
анализами:}\label{ux431ux44bux43b-ux432ux44bux434ux430ux43d-ux432ux430ux440ux438ux430ux43dux442-ux441ux43e-ux441ux43bux435ux434ux443ux44eux449ux438ux43cux438-ux430ux43dux430ux43bux438ux437ux430ux43cux438}}

\begin{enumerate}
\def\labelenumi{\arabic{enumi}.}
\tightlist
\item
  Разведочный
\item
  Причинно-следственный
\end{enumerate}

\hypertarget{ux440ux430ux437ux432ux435ux434ux43eux447ux43dux44bux439-ux430ux43dux430ux43bux438ux437}{%
\subparagraph{\texorpdfstring{\textbf{Разведочный
анализ}}{Разведочный анализ}}\label{ux440ux430ux437ux432ux435ux434ux43eux447ux43dux44bux439-ux430ux43dux430ux43bux438ux437}}

\begin{itemize}
\tightlist
\item
  В качестве примера была рассмотрена
  \href{https://cyberleninka.ru/article/n/obrabotka-i-razvedochnyy-analiz-chislovyh-massivov-dannyh}{данная
  статья}.
\end{itemize}

Разведочный анализ зачастую связывают с нахождением общих
закономерностей, распределений и аномалий, а также построением начальных
моделей, зачастую с использованием инструментов визуализации. При
выполнении данный действий мы выявляем в данных значимо отличающиеся
значения (выбросы).

В данной статье как раз и применяется с помощью критериев Диксона,
r-статистики, Граббса и других. К примеру в алгоритме с критерием
Диксона:

\begin{quote}
\ldots после ранжирования исходной выборки в порядке не убывания
рассчитывают фактическое значение критерия.
\end{quote}

А фактическое значение критерия высчитывается с помощью следующей
формул:

\begin{itemize}
\tightlist
\item
  \(r_{n} = \frac{x_{n} - \tilde{x}}{D_{x}\sqrt{\frac{n-1}{n}}}\) - при
  подозрении \(x_{n}\) на аномальный максимум;
\item
  \(r_{1} = \frac{\tilde{x} - x_{1}}{D_{x}\sqrt{\frac{n-1}{n}}}\) - при
  подозрении \(x_{1}\) на аномальный минимум.
\end{itemize}

В данных формулах:

\begin{itemize}
\tightlist
\item
  \(\tilde{x}\) - среднее по выборке (с учётом анализируемого значения);
\item
  \(D_{x}\) - дисперсия выборки (с учётом анализируемого значения).
\end{itemize}

\begin{center}\rule{0.5\linewidth}{0.5pt}\end{center}

\hypertarget{ux43fux440ux438ux447ux438ux43dux43dux43e-ux441ux43bux435ux434ux441ux442ux432ux435ux43dux43dux44bux439-ux430ux43dux430ux43bux438ux437}{%
\subparagraph{\texorpdfstring{\textbf{Причинно-следственный
анализ}}{Причинно-следственный анализ}}\label{ux43fux440ux438ux447ux438ux43dux43dux43e-ux441ux43bux435ux434ux441ux442ux432ux435ux43dux43dux44bux439-ux430ux43dux430ux43bux438ux437}}

\begin{itemize}
\tightlist
\item
  В качестве примера была рассмотрена
  \href{https://elibrary.ru/download/elibrary_35222055_16840035.pdf}{данная
  статья}.
\end{itemize}

Анализ причинно-следственных связей -- это структурированный метод,
применяемый для определения возможных причин нежелательного события или
проблемы. Он систематизирует возможные влияющие факторы в обобщенные
категории таким образом, что позволяет рассматривать все возможные
гипотезы.

В данной статье причинно-следственный метод применяется к такой операции
как поверка газовых счетчиков

Основная суть метода раскрывается в причинно-следственной диаграмме

\includegraphics{955.png}

Причём во время Механистического моделирования использовалась
приведённая ниже небольшая часть таблицы с массивом данных расчёта
забойного давления по методу Анзари и формуле Адамова.

\end{document}
